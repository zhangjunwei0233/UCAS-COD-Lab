% 这是中国科学院大学计算机科学与技术专业《计算机组成原理(研讨课)》使用的实验报告 Latex 模板
% 本模板与 2024 年 2 月 Jun-xiong Ji 完成, 更改自由 Shing-Ho Lin 和 Jun-Xiong Ji 于 2022 年 9 月共同完成的基础物理实验模板
% 如有任何问题, 请联系: jijunxoing21@mails.ucas.ac.cn
% This is the LaTeX template for report of Experiment of Computer Organization and Design courses, based on its provided Word template. 
% This template is completed on Febrary 2024, based on the joint collabration of Shing-Ho Lin and Junxiong Ji in September 2022. 
% Adding numerous pictures and equations leads to unsatisfying experience in Word. Therefore LaTeX is better. 
% Feel free to contact me via: jijunxoing21@mails.ucas.ac.cn

\documentclass[11pt]{article}

\usepackage[a4paper]{geometry}
\geometry{left=2.0cm,right=2.0cm,top=2.5cm,bottom=2.5cm}

\usepackage{ctex} % 支持中文的LaTeX宏包
\usepackage{amsmath,amsfonts,graphicx,subfigure,amssymb,bm,amsthm,mathrsfs,mathtools,breqn} % 数学公式和符号的宏包集合
\usepackage{algorithm,algorithmicx} % 算法和伪代码
\usepackage[noend]{algpseudocode} % 算法和伪代码
\usepackage{fancyhdr} % 自定义页眉页脚
\usepackage[framemethod=TikZ]{mdframed} % 创建带边框的框架
\usepackage{fontspec} % 字体设置
\usepackage{adjustbox} % 调整盒子大小
\usepackage{fontsize} % 设置字体大小
\usepackage{tikz,xcolor} % 绘制图形和使用颜色
\usepackage{multicol} % 多栏排版
\usepackage{multirow} % 表格中合并单元格
\usepackage{pdfpages} % 插入PDF文件
\usepackage{listings} % 在文档中插入源代码
\usepackage{wrapfig} % 文字绕排图片
\usepackage{bigstrut,multirow,rotating} % 支持在表格中使用特殊命令
\usepackage{booktabs} % 创建美观的表格
\usepackage{circuitikz} % 绘制电路图
\usepackage{zhnumber} % 中文序号(用于标题)
\usepackage{tabularx} % 表格折行

\definecolor{dkgreen}{rgb}{0,0.6,0}
\definecolor{gray}{rgb}{0.5,0.5,0.5}
\definecolor{mauve}{rgb}{0.58,0,0.82}
\lstset{
  frame=tb,
  aboveskip=3mm,
  belowskip=3mm,
  showstringspaces=false,
  columns=flexible,
  framerule=1pt,
  rulecolor=\color{gray!35},
  backgroundcolor=\color{gray!5},
  basicstyle={\small\ttfamily},
  numbers=none,
  numberstyle=\tiny\color{gray},
  keywordstyle=\color{blue},
  commentstyle=\color{dkgreen},
  stringstyle=\color{mauve},
  breaklines=true,
  breakatwhitespace=true,
  tabsize=3,
}

% 轻松引用, 可以用\cref{}指令直接引用, 自动加前缀. 
% 例: 图片label为fig:1
% \cref{fig:1} => Figure.1
% \ref{fig:1}  => 1
\usepackage[capitalize]{cleveref}
% \crefname{section}{Sec.}{Secs.}
\Crefname{section}{Section}{Sections}
\Crefname{table}{Table}{Tables}
\crefname{table}{Table.}{Tabs.}

% \setmainfont{Palatino Linotype.ttf}
% \setCJKmainfont{SimHei.ttf}
% \setCJKsansfont{Songti.ttf}
% \setCJKmonofont{SimSun.ttf}
\punctstyle{kaiming}
% 偏好的几个字体, 可以根据需要自行加入字体ttf文件并调用

\renewcommand{\emph}[1]{\begin{kaishu}#1\end{kaishu}}

% 对 section 等环境的序号使用中文
\renewcommand \thesection{\zhnum{section}、}
\renewcommand \thesubsection{\arabic{section}}


%%%%%%%%%%%%%%%%%%%%%%%%%%%
%改这里可以修改实验报告表头的信息
\newcommand{\name}{张钧玮}
\newcommand{\studentNum}{2023K8009908003}
\newcommand{\major}{计算机科学与技术}
\newcommand{\labNum}{5.2}
\newcommand{\labName}{高速缓存(Cache)设计}
%%%%%%%%%%%%%%%%%%%%%%%%%%%

\begin{document}

\input{tex_file/head.tex}

\section{icache的逻辑电路结构与仿真波形的截图及说明}

\noindent
$\bullet$
\textbf{电路结构与工作原理}

\begin{figure}[H]
  \centering
  \includegraphics[width=10cm]{fig/prj5.2/icache_structure.png}
  \caption{icache电路结构图(只画出一路)}
  \label{fig:icache_structure}
\end{figure}

\noindent
关键设计点说明:

指令缓存采用4路-组相连设计,共有四路(Way),每一路八个集合(Set),每一个集合存放一个相邻的数据块(Block),大小为8字(word),每个字长4字节。

接收到CPU的访存地址之后,cache将地址拆分为三部分:Tag, Index, Offset。

1. Offset: 从一个Block中选择出一个word

2. Index:每一个index对应一个Set

3. Tag:根据Tag从同一个Set中的各个Way中判断数据命中情况

cache表现出的行为为:根据CPU首先找出4Way中Index像对应的Set,这会选中4个Block。然后判断有无Tag与请求Tag相同的Block。若有,则缓存命中,根据Offest从该Block中选中对应的word返回;若没有,则缓存未命中,此时需要从这四个Block中选中一个替换,然后再按命中的逻辑选择并返回word。


\vspace{1ex}

\noindent
$\bullet$
\textbf{替换策略选择}

\noindent
关键设计点说明:

有很多可选择的替换策略:

1. 随机替换:适用于内存读写位置随机的系统(即不满足\textbf{局部性原理})

2. 最近最少使用(LRU):维护每个Set中最早一个被使用的一个Block,需要时替换该Block(符合\textbf{局部性原理})

3. 最不常用(LFU):维护每个Set中每个Block的命中次数,需要时替换次数最小的Block。(在实际应用中效率低,较少使用)

4. 最近最多使用(MRU):维护每个Set中最近一个被使用的一个Block,需要时替换该Block(适用于旧数据会被新数据更多访问的系统)

这里,由于CPU的L1指令缓存满足局部性原理,因此使用LRU算法较为合适。但是由于完整地记录最早被使用的Block需要较多资源,所以我采用了\textbf{使用二叉树近似的伪LRU算法}:

每一个Set有4路Block,维护一个二叉树,用于记录最近一次被使用的Block的路径(0节点表示向左走,1节点表示向右走)。每次访存,更新被访问的Block的路径到二叉树中。当需要替换的时候,按照相反的方向走(0向右,1向左),即可找到近似的最近最少使用Block。

\noindent
$\bullet$
\textbf{核心代码实现}

\begin{figure}[H]
  \centering
  \includegraphics[width=8cm]{fig/prj5.2/icache_FSM.png}
  \caption{icache状态转移代码}
  \label{fig:icache_FSM}
\end{figure}

按照讲义的状态转移图编写,不做过多解释。

\begin{figure}[H]
  \centering
  \includegraphics[width=10cm]{fig/prj5.2/icache_body.png}
  \caption{icache主体代码}
  \label{fig:icache_body}
\end{figure}

WAIT阶段等待CPU的ready信号拉高,并将请求地址保存进寄存器中

TAG\_RD阶段判断CPU请求是否命中

若未命中,则向内存发送读请求,并将Burst传输得到的Block存放入寄存器中,进而写入cache中的替换目标Block,最后回应CPU请求

若命中,直接读出数据并回应CPU请求

\begin{figure}[H]
  \centering
  \includegraphics[width=10cm]{fig/prj5.2/icache_evict.png}
  \caption{icache替换算法代码}
  \label{fig:icache_evict}
\end{figure}

为每一个Set设置一个三位的二叉树LRU[2:0]。初始化为000,之后每次访存都需要跟新树的路径。从LRU可以时时产生leastUsedWay信号。

替换时,如果有空闲block,则evictWay有限设置为firstEmptyWay,否则设置为leastUsedWay

\section{dcache的逻辑电路结构与仿真波形的截图及说明}

\noindent
$\bullet$
\textbf{状态转移设计}

\begin{figure}[H]
  \centering
  \includegraphics[width=10cm]{fig/prj5.2/dcache_design.png}
  \caption{dcache状态转移图}
  \label{fig:dcache_design}
\end{figure}

图中,绿色为缓存命中情况,红色为未命中情况,黄色为数据旁路。

数据缓存的写命中策略选择Write-back而不是直接写入内存,写缺失策略选择Write-allocate对写地址附近的数据进行缓存。这两种策略通常结合使用,用于处理CPU运行时的有大量内存区域可缓存,且满足局部性原理的情况。

数据缓存的核心结构和替换算法与指令缓存类似,不多赘述。

\noindent
$\bullet$
\textbf{核心代码实现}

\begin{figure}[H]
  \centering
  \includegraphics[width=10cm]{fig/prj5.2/dcache_wait.png}
  \caption{dcache等待CPU请求}
\end{figure}

等待接受CPU请求,将接收到的请求存入寄存器。

需要注意,如果CPU发送旁路的写请求,说明写入地址不可缓存,因此需要等待数据真正写入内存中后再拉高ready信号。对于其他情况,可以直接拉高ready信号。

\begin{figure}[H]
  \centering
  \includegraphics[width=10cm]{fig/prj5.2/icache_tagrd.png}
  \caption{dcache检查CPU请求命中情况}
\end{figure}

分析CPU请求,具体包括:是读还是写,是否触发旁路,缓存是否命中。并根据结果确定下一状态

\begin{figure}[H]
  \centering
  \includegraphics[width=10cm]{fig/prj5.2/dcache_rdcch.png}
  \caption{dcache读取命中数据}
\end{figure}

若读请求缓存命中,则生成命中数据

\begin{figure}[H]
  \centering
  \includegraphics[width=10cm]{fig/prj5.2/dcache_rsp.png}
  \caption{dcache回应CPU请求}
\end{figure}

回应CPU的读请求,需要处理缓存和旁路两种情况

\begin{figure}[H]
  \centering
  \includegraphics[width=10cm]{fig/prj5.2/dcache_eviction.png}
  \caption{dcache数据替换}
\end{figure}

在内存写请求命中时,需要将缓存中的数据读出,写入CPU请求数据后重新写回缓存

在缓存未命中时,需要将从内存中读出的请求地址附近的数据(如果是写请求,还需要进行写入)写入替换目标Block

\begin{figure}[H]
  \centering
  \includegraphics[width=10cm]{fig/prj5.2/dcache_memwreq.png}
  \caption{dcache发送memWrite请求}
\end{figure}

\begin{figure}[H]
  \centering
  \includegraphics[width=10cm]{fig/prj5.2/dcache_memrreq.png}
  \caption{dcache发送memRead请求}
\end{figure}

在缓存未命中或者触发旁路的情况下,缓存需要向内存发送读写请求,并根据这两种情况的不同设置对应的len信号

\begin{figure}[H]
  \centering
  \includegraphics[width=10cm]{fig/prj5.2/dcache_memwrsp.png}
  \caption{dcache发送内存写数据}
\end{figure}

缓存在发送内存写请求后,需要用Burst传输的方式向内存传递写数据。

使用一个Shifter控制Burst数据的传输与last信号的产生。根据旁路与内存两种情况设置shifter的对应位位1,之后每传送一个word右移一位,直至最低位为1时产生last信号结束传输。

\begin{figure}[H]
  \centering
  \includegraphics[width=10cm]{fig/prj5.2/dcache_memrrsp.png}
  \caption{dcache接受内存读数据}
\end{figure}

缓存在发送内存写请求后,用Burst传输的方式接受内存数据并存入寄存器。

\begin{figure}[H]
  \centering
  \includegraphics[width=10cm]{fig/prj5.2/dcache_evict.png}
  \caption{dcache替换算法}
\end{figure}

与指令缓存类似,不多赘述。

\section{cache的性能分析(CPU为多周期CPU,还未实现流水线)}

\noindent
$\bullet$
\textbf{microbench性能分析}

\begin{figure}[H]
  \centering
  \includegraphics[width=\textwidth]{fig/prj5.2/perf_analysis.png}
  \caption{microbench性能分析(单位: ms)}
\end{figure}

从图中可以看出,在实现icache之后,所有的microbench(除了Eratosthenes sieve)均得到了大幅度的提升,很多提升甚至达到数量级级别。这是因为所有的指令都需要取指这一环节,icache优化了取指时间,故而对所有的testbench都带来了提升

在icache的基础上进一步实现dcache之后,发现访存指令占比较高的micorbench(如Dinic)得到了明显的提升,而访存指令较少的microbench(如Fibonacci)提升较为有限。总体而言,dcache带来了很好的性能提升

\noindent
$\bullet$
\textbf{turbo_eval性能分析}

1. 无cache情况:CoreMark得分84376120, Dhrystone得分483

2. 实现icache之后:CoreMark得分8528206, Dhrystone得分3371

3. 进一步实现dcache之后:CoreMark得分8318327, Dhrystone得分4657

由此可见,最初时icache为瓶颈。实现icache之后跑分得到了数量级的提升。然而在进一步实现dcache之后跑分的提升却比较有限,推测此时的瓶颈并不在dcache,有可能在流水线或者其他部分。

\noindent
$\bullet$
\textbf{dcache替换策略选择}

一开始时沿用icache的策略,选择优先替换空Block,之后替换最近最少使用Block。但是在测试时想到dcache比icache多了一个dirty信息,是否可以用其进行性能优化。

为此,考虑将替换策略修改为:优先替换空Block,之后替换非Dirty的Block(因为此时替换不需要内存写回(Write back)),最后选择最近最少使用Block

这样做虽然减小了写回用时,但是也会造成额外的开销:非Dirty的Block虽然没有被写入数据,但是有可能被频繁的读取。如果每次替换都将其替换掉,则会造成Dirty Block在Cache中堆积,产生很多的Miss。事实证明却是如此,替换为此策略之后跑分均有所下降,故最后修改回了最初的策略。

\section{实验过程中遇到的问题、对问题的思考过程及解决方法}

\noindent
$\bullet$
\textbf{RTL设计问题}。

在第一次实现dcache的时候,我在WAIT阶段就将ready信号统一拉高,但发现运行测试用例的时在程序的最后一条指令处报错,通过查看反汇编结果发现程序执行了trap中的指令,而golden CPU并没有落入trap中而是直接返回,导致了CPU行为不一致。

这是因为程序在trap之前的最后一条指令是通过旁路向内存中写入数据触发系统调用,但是在原先的cache实现中,CPU发送写请求后握手后就继续执行接下来的指令,但是此时cache仅仅接收到旁路写请求,还没有真正写入数据。

因此,需要修改旁路写的ready信号。当cache接收到旁路写请求的时候,不直接拉高ready信号,而是等待真正把数据接入内存之后,再拉高ready信号。


\section{对讲义中思考题(如有)的理解和回答}


本次实验没有思考题

\section{收获与感想}

通过本次高速缓存设计实验,我对计算机存储体系的核心机制有了更深入的理解。在实现指令缓存(icache)时,采用4路组相连结构配合二叉树近似的伪LRU替换算法,验证了组相连映射在空间利用率与访问延迟间的平衡优势。

此外,我通过实现Write-back写回和Write-allocate写分配策略,理解了缓存一致性的维护机制。特别在旁路写入(bypass)场景中,发现必须确保数据真正写入内存后才能响应CPU的ready信号,否则会导致后续指令错误执行。此问题的解决过程强化了对存储器时序安全性的认识。

在调试过程中,通过波形分析验证了状态机的正确跳转:如TAG\_RD阶段的命中判断、EVICTION阶段的脏块回写、MEM\_WRSP阶段的burst传输控制等关键节点。这提升了使用Verilog进行复杂状态机设计的能力,并体会到缓存性能对整体系统的影响——合理的块大小(8字)和路数(4路)选择可显著提升局部性访问效率。

\section{实验所耗时间}

\newcommand{\labTime}{12} % 请在这里填写实验所耗时间(单位:小时)

在课后,你花费了大约\underline{\makebox[5em][c]{\labTime}}小时完成此次实验。

\end{document}