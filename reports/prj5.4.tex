% 这是中国科学院大学计算机科学与技术专业《计算机组成原理(研讨课)》使用的实验报告 Latex 模板
% 本模板与 2024 年 2 月 Jun-xiong Ji 完成, 更改自由 Shing-Ho Lin 和 Jun-Xiong Ji 于 2022 年 9 月共同完成的基础物理实验模板
% 如有任何问题, 请联系: jijunxoing21@mails.ucas.ac.cn
% This is the LaTeX template for report of Experiment of Computer Organization and Design courses, based on its provided Word template. 
% This template is completed on Febrary 2024, based on the joint collabration of Shing-Ho Lin and Junxiong Ji in September 2022. 
% Adding numerous pictures and equations leads to unsatisfying experience in Word. Therefore LaTeX is better. 
% Feel free to contact me via: jijunxoing21@mails.ucas.ac.cn

\documentclass[11pt]{article}

\usepackage[a4paper]{geometry}
\geometry{left=2.0cm,right=2.0cm,top=2.5cm,bottom=2.5cm}

\usepackage{ctex} % 支持中文的LaTeX宏包
\usepackage{amsmath,amsfonts,graphicx,subfigure,amssymb,bm,amsthm,mathrsfs,mathtools,breqn} % 数学公式和符号的宏包集合
\usepackage{algorithm,algorithmicx} % 算法和伪代码
\usepackage[noend]{algpseudocode} % 算法和伪代码
\usepackage{fancyhdr} % 自定义页眉页脚
\usepackage[framemethod=TikZ]{mdframed} % 创建带边框的框架
\usepackage{fontspec} % 字体设置
\usepackage{adjustbox} % 调整盒子大小
\usepackage{fontsize} % 设置字体大小
\usepackage{tikz,xcolor} % 绘制图形和使用颜色
\usepackage{multicol} % 多栏排版
\usepackage{multirow} % 表格中合并单元格
\usepackage{pdfpages} % 插入PDF文件
\usepackage{listings} % 在文档中插入源代码
\usepackage{wrapfig} % 文字绕排图片
\usepackage{bigstrut,multirow,rotating} % 支持在表格中使用特殊命令
\usepackage{booktabs} % 创建美观的表格
\usepackage{circuitikz} % 绘制电路图
\usepackage{zhnumber} % 中文序号(用于标题)
\usepackage{tabularx} % 表格折行

\definecolor{dkgreen}{rgb}{0,0.6,0}
\definecolor{gray}{rgb}{0.5,0.5,0.5}
\definecolor{mauve}{rgb}{0.58,0,0.82}
\lstset{
  frame=tb,
  aboveskip=3mm,
  belowskip=3mm,
  showstringspaces=false,
  columns=flexible,
  framerule=1pt,
  rulecolor=\color{gray!35},
  backgroundcolor=\color{gray!5},
  basicstyle={\small\ttfamily},
  numbers=none,
  numberstyle=\tiny\color{gray},
  keywordstyle=\color{blue},
  commentstyle=\color{dkgreen},
  stringstyle=\color{mauve},
  breaklines=true,
  breakatwhitespace=true,
  tabsize=3,
}

% 轻松引用, 可以用\cref{}指令直接引用, 自动加前缀. 
% 例: 图片label为fig:1
% \cref{fig:1} => Figure.1
% \ref{fig:1}  => 1
\usepackage[capitalize]{cleveref}
% \crefname{section}{Sec.}{Secs.}
\Crefname{section}{Section}{Sections}
\Crefname{table}{Table}{Tables}
\crefname{table}{Table.}{Tabs.}

% \setmainfont{Palatino Linotype.ttf}
% \setCJKmainfont{SimHei.ttf}
% \setCJKsansfont{Songti.ttf}
% \setCJKmonofont{SimSun.ttf}
\punctstyle{kaiming}
% 偏好的几个字体, 可以根据需要自行加入字体ttf文件并调用

\renewcommand{\emph}[1]{\begin{kaishu}#1\end{kaishu}}

% 对 section 等环境的序号使用中文
\renewcommand \thesection{\zhnum{section}、}
\renewcommand \thesubsection{\arabic{section}}


%%%%%%%%%%%%%%%%%%%%%%%%%%%
%改这里可以修改实验报告表头的信息
\newcommand{\name}{张钧玮}
\newcommand{\studentNum}{2023K8009908003}
\newcommand{\major}{计算机科学与技术}
\newcommand{\labNum}{5.4}
\newcommand{\labName}{DMA引擎和中断处理}
%%%%%%%%%%%%%%%%%%%%%%%%%%%

\begin{document}

\input{tex_file/head.tex}

\section{DMA引擎的设计}

\noindent
$\bullet$
\textbf{状态及状态转移设计}

\begin{lstlisting}[language=Verilog] % 指定语言
	/* state encodings */
	localparam RD_IDLE = 3'b001,
	           RD_REQ  = 3'b010,
	           RD      = 3'b100,
	           WR_IDLE = 3'b001,
	           WR_REQ  = 3'b010,
	           WR      = 3'b100;
	reg [2:0] rdCurrentState, rdNextState;
	reg [2:0] wrCurrentState, wrNextState;
    
    /* FSM for read */
	always @(posedge clk) begin
		if (rst)
			rdCurrentState <= RD_IDLE;
		else
			rdCurrentState <= rdNextState;
	end
	always @(*) begin
		case (rdCurrentState)
			RD_IDLE: begin
				if (en & (wrCurrentState == WR_IDLE) & headPtrReg != tailPtrReg)
					rdNextState = RD_REQ;
				else
					rdNextState = RD_IDLE;
			end
			RD_REQ: begin
				if (rd_req_ready & rd_req_valid)
					rdNextState = RD;
				else if (rdBurstCnt == nrBurst) // finished reading whole dma-size
					rdNextState = RD_IDLE;
				else
					rdNextState = RD_REQ;
			end
			RD: begin
				if (rd_valid & rd_ready & rd_last)
					rdNextState = RD_REQ;
				else
					rdNextState = RD;
			end
			default:
				rdNextState = RD_IDLE;
		endcase
	end

	/* FSM for write */
	always @(posedge clk) begin
		if (rst)
			wrCurrentState <= WR_IDLE;
		else
			wrCurrentState <= wrNextState;
	end
	always @(*) begin
		case (wrCurrentState)
			WR_IDLE: begin
				if (en & headPtrReg != tailPtrReg & !fifo_is_empty) // no need to wait for reading to end, start as long as fifo non-empty
					wrNextState = WR_REQ;
				else
					wrNextState = WR_IDLE;
			end
			WR_REQ: begin
				if (wr_req_ready & wr_req_valid)
					wrNextState = WR;
				else if (wrBurstCnt == nrBurst)
					wrNextState = WR_IDLE;
				else
					wrNextState = WR_REQ;
			end
			WR: begin
				if (wr_valid & wr_ready & wr_last)
					wrNextState = WR_REQ;
				else
					wrNextState = WR;
			end
			default:
				wrNextState = WR_IDLE;
		endcase
	end
\end{lstlisting}

DMA引擎由两个独立的状态机分别控制读和写。其中读引擎在DMA启动(en位为1)且尾指针落后于头指针的时候开始从buffer0读取数据。通过burst传输的方式将一整个子buffer的数据(dma\_size)分多次读取到FIFO中。

在每次burst传输读取数据到FIFO后,写引擎随即启动,从FIFO中读出数据并将数据写到buffer1中的对应位置。

\noindent
$\bullet$
\textbf{与CPU的交互}

\begin{lstlisting}[language=Verilog]
    /* control registers */
	reg [31:0] srcBaseReg;
	reg [31:0] destBaseReg;
	reg [31:0] tailPtrReg;
	reg [31:0] headPtrReg;
	reg [31:0] dmaSizeReg;
	reg [31:0] ctrlStatReg;
	// parse the ctrlStatReg
	assign intr = ctrlStatReg[31];
	wire   en   = ctrlStatReg[0];
    
    /* interact with CPU */
	always @(posedge clk) begin
		if (rst) begin
			srcBaseReg  <= 32'b0;
			destBaseReg <= 32'b0;
			tailPtrReg  <= 32'b0;
			headPtrReg  <= 32'b0;
			dmaSizeReg  <= 32'b0;
			ctrlStatReg <= 32'b0;
		end else if (en & transmitFinish) begin // send intr to CPU
			tailPtrReg <= tailPtrReg + dmaSizeReg;
			ctrlStatReg[31] <= 1'b1;
		end else begin // receive control from CPU
			if (reg_wr_en[0])  srcBaseReg  <= reg_wr_data;
			if (reg_wr_en[1])  destBaseReg <= reg_wr_data;
			if (reg_wr_en[2])  tailPtrReg  <= reg_wr_data;
			if (reg_wr_en[3])  headPtrReg  <= reg_wr_data;
			if (reg_wr_en[4])  dmaSizeReg  <= reg_wr_data;
			if (reg_wr_en[5])  ctrlStatReg <= reg_wr_data;
		end
	end
\end{lstlisting}

DMA引擎通过一些状态寄存器记录当前的工作状态,这些寄存器为CPU可见的。CPU可以随时将数据写入这些寄存器,用于更新头指针,交代dma\_size等等。

在每次一整个子buffer的数据传输完毕后,DMA引擎需要拉高INTR位的信号通知CPU,触发中断。

\noindent
$\bullet$
\textbf{读写次数与长度控制}

\begin{lstlisting}[language=Verilog]
    /* burst count control */
	reg  [31:0] rdBurstCnt, wrBurstCnt; // keep track of progress
	// each burst transmits 32 bytes, encoded by the last 5 bits
	wire [31:0] nrFullBurst = {5'b0, dmaSizeReg[31:5]};
	wire [31:0] nrPartBurst = {31'b0, |dmaSizeReg[4:0]}; // 0 or 1
	wire [31:0] nrBurst     = nrFullBurst + nrPartBurst;
	// rdBurstCnt and wrBurstCnt counts from 0 to nrBurst-1, and finished while
	// they are equal
	wire transmitFinish = (wrCurrentState == WR_REQ) & (wrBurstCnt == nrBurst) & (rdBurstCnt == nrBurst);
	// update rdBurstCnt and wrBurstCnt
	always @(posedge clk) begin
		if (rst)
			rdBurstCnt <= 32'b0;
		else if (rdCurrentState == RD_IDLE & wrCurrentState == WR_IDLE)
			rdBurstCnt <= 32'b0;
		else begin // accumulate
			if (rdCurrentState == RD & rd_valid & rd_ready & rd_last)
				rdBurstCnt <= rdBurstCnt + 32'b1;
			else
				rdBurstCnt <= rdBurstCnt;
		end
	end
	always @(posedge clk) begin
		if (rst)
			wrBurstCnt <= 32'b0;
		else if (rdCurrentState == RD_IDLE & wrCurrentState == WR_IDLE)
			wrBurstCnt <= 32'b0;
		else begin // accumulate
			if (wrCurrentState == WR & wr_ready & wr_valid & wr_last)
				wrBurstCnt <= wrBurstCnt + 32'b1;
			else
				wrBurstCnt <= wrBurstCnt;
		end
	end

	/* burst length control */
	wire [4:0] nrFullWords = {2'b0, dmaSizeReg[4:2]};
	wire [4:0] nrFracWords = {4'b0, |dmaSizeReg[1:0]};
	wire [4:0] lastBurstLen = nrFullWords + nrFracWords - 5'b1; // unit: words
	assign rd_req_len = (nrPartBurst[0] & (rdBurstCnt == (nrBurst - 32'b1))) ? lastBurstLen : 5'd7;
	assign wr_req_len = (nrPartBurst[0] & (wrBurstCnt == (nrBurst - 32'b1))) ? lastBurstLen : 5'd7;
\end{lstlisting}

一个子buffer的传输任务需要分为多次burst传输,因此需要控制传输的次数及每次传输的长度。

每次执行传输任务时根据dma\_size计算得到需要的burst传输次数,并在传输时分别维护一个读写次数计数器,从0开始,每次burst传输后计数器加1,当计数器的数值等于需要的传输次数时,传输终止。

除了最后一次burst传输,其他的burst传输长度均为最长长度32字节。最后一次传输的长度根据dma\_size的32字节偏移(后5位)计算得到。

\noindent
$\bullet$
\textbf{内存与FIFO数据读写控制}

\begin{lstlisting}[language=Verilog]
    /* read request */
	assign rd_req_valid = (rdCurrentState == RD_REQ) & ~fifo_is_full & (rdBurstCnt != nrBurst);
	reg [31:0] readAddrReg;
	always @(posedge clk) begin
		if (rdCurrentState == RD_IDLE & wrCurrentState == WR_IDLE & headPtrReg != tailPtrReg) // init addr after IDLE
			readAddrReg <= srcBaseReg + tailPtrReg;
		else if ((rdCurrentState == RD) & rd_ready & rd_valid & rd_last) begin // update addr after each burst
			if (rdBurstCnt == (nrBurst - 32'b1)) // the last burst
				readAddrReg <= readAddrReg + {27'b0, dmaSizeReg[4:0]};
			else
				readAddrReg <= readAddrReg + 32'd32;
		end
	end
	assign rd_req_addr = readAddrReg;

	/* read rsp */
	assign rd_ready = (rdCurrentState == RD) & ~fifo_is_full;
	// store the data to FIFO
	assign fifo_wen = rd_ready & rd_valid;
	assign fifo_wdata = rd_rdata;

	/* write request */
	assign wr_req_valid = (wrCurrentState == WR_REQ) & ~fifo_is_empty & (wrBurstCnt != nrBurst);
	reg [31:0] writeAddrReg;
	always @(posedge clk) begin
		if (rdCurrentState == RD_IDLE & wrCurrentState == WR_IDLE & headPtrReg != tailPtrReg) // init addr after IDLE
			writeAddrReg <= destBaseReg + tailPtrReg;
		else if ((wrCurrentState == WR) & wr_ready & wr_valid & wr_last) begin // update addr after each burst
			if (wrBurstCnt == (nrBurst - 32'd1))
				writeAddrReg <= writeAddrReg + {27'b0, dmaSizeReg[4:0]};
			else
				writeAddrReg <= writeAddrReg + 32'd32;
		end
	end
	assign wr_req_addr = writeAddrReg;

	/* write rsp */
	// read data from FIFO, data will return in the next clk cycle!
	assign fifo_rden = (wrCurrentState == WR) & ~fifo_is_empty & wr_ready;

	reg wrValidReg; // delay one clk cycle
	always @(posedge clk) begin
		wrValidReg <= fifo_rden;
	end
	assign wr_valid = wrValidReg;

	assign wr_data = fifo_rdata; // FIFO will hold the data, so no need for registration

	// generate last signal
	reg [7:0] lastGenerator;
	always @(posedge clk) begin
		if (wr_req_ready & wr_req_valid) // init lastGenerator when entering WR state
			lastGenerator <= (8'b1 << (wr_req_len + 1));
		else if (fifo_rden) // shift whenever fifo is read
			lastGenerator <= {1'b0, lastGenerator[7:1]};
		else
			lastGenerator <= lastGenerator;
	end
	assign wr_last = lastGenerator[0];
\end{lstlisting}

在一次读burst传输中,DMA引擎首先向内存发送读请求,然后按burst传输的方式接受数据并将其存入FIFO中。

随后在写burst传输中,DMA引擎向内存发出写请求,然后按burst传输的方式从FIFO中读出数据并发送给内存。期间需要自己维护一个shifter,用于产生last信号。

\section{支持中断处理的MIPS CPU设计}

在原先的基础上增加INTR状态编码与相应的状态转移

\begin{lstlisting}[language=Verilog]
	localparam INIT = 10'b0000000001,
	           IF   = 10'b0000000010,
	           IW   = 10'b0000000100,
	           ID   = 10'b0000001000,
	           EX   = 10'b0000010000,
	           ST   = 10'b0000100000,
	           WB   = 10'b0001000000,
	           LD   = 10'b0010000000,
	           RDW  = 10'b0100000000,
	           INTR = 10'b1000000000;
    
    // ...
    
    wire is_ERET         = opcode == 6'b010000 & func == 6'b011000;
    
    // ...
    
    IF: begin
			if (passedIntr)
				next_state = INTR;
			else if (Inst_Req_Ready)
				next_state = IW;
			else
				next_state = IF;
		end
	EX: begin
				if (is_REGIMMType | is_IBranchType | is_j | is_ERET)
					next_state = IF;
		// ...
	INTR: begin
		next_state = IF;
	end

		
	// ...
	
	assign Inst_Req_Valid  = (current_state == IF & ~passedIntr)           ? 1'b1 : 1'b0;
\end{lstlisting}

在IF状态下,如果接收到中断信号,则进入INTR状态。

在EX阶段,如果当前指令位ERET,则从中断中返回。

在INTR状态下:保存当前PC到EPC,然后更新PC到中断程序入口,退出时恢复

\begin{lstlisting}[language=Verilog]
    /* interruption manage */
	reg [31:0] EPC;
	always @(posedge clk) begin
		if (rst)
			EPC <= 32'b0;
		else if (current_state == INTR)
			EPC <= pcReg; // store current PC
		else
			EPC <= EPC;
	end

    // in PC update logic
    else if (current_state == INTR)
		pcReg <= 32'h100; // entrance of intr program
	else if (current_state == EX & is_ERET)
		pcReg <= EPC; // restore EPC to PC
	// ...

\end{lstlisting}

在中断程序中,需要屏蔽其他的中断信号:

\begin{lstlisting}[language=Verilog]
    // interupt
	reg blockIntr;
	wire passedIntr = intr & (~blockIntr);

    /* interruption blocking */
	always @(posedge clk) begin
		if (rst)
			blockIntr <= 1'b0;
		else if (current_state == INTR)
			blockIntr <= 1'b1;
		else if (current_state == EX & is_ERET)
			blockIntr <= 1'b0;
		else
			blockIntr <= blockIntr;
	end
\end{lstlisting}

\section{中断汇编程序设计}

\begin{lstlisting}[language=Asm]
# reset INTR bit of ctrl_stat to 0
	la $k0, 0x60020000    # dma_mmio base addr (given in P04)
	lw $k0, 0x14($k0)     # load ctrl_stat to k0, with offset 0x14
	li $k1, 0x7fffffff
	and $k0, $k1, $k0     # set INTR bit to 0
	la $k1, 0x60020000
	sw $k0, 0x14($k1)     # store ctrl_stat back

	# calculate new_tail_ptr - old_tail_ptr
	la $k1, 0x60020000
	lw $k1, 0x8($k1)      # load new_tail_ptr to k1
	la $k0, last_tail_ptr
	lw $k0, 0($k0)        # load last_tail_ptr to k0
	sub $k0, $k1, $k0     # save new - old to k0

	# decrease dma_buf_stat by (new - old) / dma_size
	b $TestCondition
	nop
$Loop:
	lw $k1, 0x10($0)        # load dma_buf_stat to k1
	addi $k1, $k1, -1     # dma_buf_stat--
	sw $k1, 0x10($0)      # save dma_buf_stat back

	la $k1, 0x60020000
	lw $k1, 0x10($k1)     # load dma_size to k1
	sub $k0, $k0, $k1     # subtract dma_size from new - old

$TestCondition:
	bgtz $k0, $Loop
	nop

	# save last_tail_ptr for future use
	la $k0, last_tail_ptr
	la $k1, 0x60020000
	lw $k1, 0x08($k1)
	sw $k1, 0($k0)

	# return with a eret
	eret
\end{lstlisting}

中断程序首先相应中断,即向ctrl\_stat中的INTR信号置0。

然后根据DMA更新的tail\_ptr和上一次的tail\_ptrb比较得出在上一次中断间DMA搬运了多少个dma\_size大小的数据快,并减到全局变量dma\_buf\_stat上。

最后,程序将当前的tail\_ptr保存留作下一次使用,然后返回。

\section{不同实现方法的性能差异}

在mover.c文件中添加程序计数器统计程序运行的时钟周期数,得到如下结果:

$\bullet$ 使用DMA时:消耗76075576周期

$\bullet$ 仅使用软件时:消耗188079554周期

由此可见,使用硬件处理数据搬运获得了2.5倍的性能提升


\section{实验过程中遇到的问题、对问题的思考过程及解决方法}

\noindent
$\bullet$
\textbf{RTL设计问题}。

1. 在最初的engine\_core设计中,在向内存发送写数据时,由于写数据需要从FIFO中读出,且FIFO是同步读(即拉高fifo\_rden后,数据会在下一个时钟周期有效),于是我在下个时钟周期将wr\_valid信号拉高,从而将数据送出。

但这样做的问题是写一个时钟周期的wr\_ready信号不一定拉高,如果wr\_valid信号持续这一个时钟周期,则有可能导致无法成功握手,造成数据流失。因此,需要在wr\_ready信号为0的时候将wr\_valid信号寄存,直至成功握手。

2. 在最初的engine\_core设计中,我将用于产生last信号的shifter寄存器设置成了8位,但是在我的实现中,这个寄存器需要九位位宽。这就导致了last信号在移位后被阶段丢失,从而使器件陷入死循环,最后提示custom\_cpu running time out报错。在修改位宽为9位后,通过了测试

总体而言,本次实验较为顺利。这是因为我吸取了上次dnn实验的教训,优先完成所有器件的设计,确保顾及每一处细节之后再开始编写代码。


\section{对讲义中思考题(如有)的理解和回答}

本次实验没有思考题

\section{收获与感想}

通过本次实验入理解了DMA引擎的工作原理。在DMA引擎的设计过程中,通过双状态机(读/写)的独立控制机制,实现了数据在内存与FIFO之间的高效搬运。状态机设计需严格遵循握手协议,尤其需关注rd\_req\_ready、rd\_valid等信号的时序协调,避免死锁或数据丢失。调试过程中发现,last信号生成逻辑的位宽设计错误曾导致状态机陷入死循环,这一问题的解决强化了对时序逻辑细节的重视。

在中断处理扩展的CPU设计中,新增INTR状态并引入EPC寄存器保存断点地址,实现了中断响应与返回(ERET指令)的完整流程。关键点在于中断屏蔽机制的设计:通过blockIntr信号确保中断程序执行期间不被重复打断。这一过程加深了对异常处理硬件支持的理解,尤其是程序计数器与特权寄存器协同工作的原理。

性能对比实验验证了硬件加速的显著优势。DMA传输相比纯软件搬运效率提升约2.5倍(周期数从188M降至76M),凸显了卸载CPU负载的实际价值。

\section{实验所耗时间}

\newcommand{\labTime}{8} % 请在这里填写实验所耗时间(单位:小时)

在课后,你花费了大约\underline{\makebox[5em][c]{\labTime}}小时完成此次实验。

\end{document}