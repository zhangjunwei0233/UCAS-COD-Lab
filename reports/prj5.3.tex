% 这是中国科学院大学计算机科学与技术专业《计算机组成原理(研讨课)》使用的实验报告 Latex 模板
% 本模板与 2024 年 2 月 Jun-xiong Ji 完成, 更改自由 Shing-Ho Lin 和 Jun-Xiong Ji 于 2022 年 9 月共同完成的基础物理实验模板
% 如有任何问题, 请联系: jijunxoing21@mails.ucas.ac.cn
% This is the LaTeX template for report of Experiment of Computer Organization and Design courses, based on its provided Word template. 
% This template is completed on Febrary 2024, based on the joint collabration of Shing-Ho Lin and Junxiong Ji in September 2022. 
% Adding numerous pictures and equations leads to unsatisfying experience in Word. Therefore LaTeX is better. 
% Feel free to contact me via: jijunxoing21@mails.ucas.ac.cn

\documentclass[11pt]{article}

\usepackage[a4paper]{geometry}
\geometry{left=2.0cm,right=2.0cm,top=2.5cm,bottom=2.5cm}

\usepackage{ctex} % 支持中文的LaTeX宏包
\usepackage{amsmath,amsfonts,graphicx,subfigure,amssymb,bm,amsthm,mathrsfs,mathtools,breqn} % 数学公式和符号的宏包集合
\usepackage{algorithm,algorithmicx} % 算法和伪代码
\usepackage[noend]{algpseudocode} % 算法和伪代码
\usepackage{fancyhdr} % 自定义页眉页脚
\usepackage[framemethod=TikZ]{mdframed} % 创建带边框的框架
\usepackage{fontspec} % 字体设置
\usepackage{adjustbox} % 调整盒子大小
\usepackage{fontsize} % 设置字体大小
\usepackage{tikz,xcolor} % 绘制图形和使用颜色
\usepackage{multicol} % 多栏排版
\usepackage{multirow} % 表格中合并单元格
\usepackage{pdfpages} % 插入PDF文件
\usepackage{listings} % 在文档中插入源代码
\usepackage{wrapfig} % 文字绕排图片
\usepackage{bigstrut,multirow,rotating} % 支持在表格中使用特殊命令
\usepackage{booktabs} % 创建美观的表格
\usepackage{circuitikz} % 绘制电路图
\usepackage{zhnumber} % 中文序号(用于标题)
\usepackage{tabularx} % 表格折行

\definecolor{dkgreen}{rgb}{0,0.6,0}
\definecolor{gray}{rgb}{0.5,0.5,0.5}
\definecolor{mauve}{rgb}{0.58,0,0.82}
\lstset{
  frame=tb,
  aboveskip=3mm,
  belowskip=3mm,
  showstringspaces=false,
  columns=flexible,
  framerule=1pt,
  rulecolor=\color{gray!35},
  backgroundcolor=\color{gray!5},
  basicstyle={\small\ttfamily},
  numbers=none,
  numberstyle=\tiny\color{gray},
  keywordstyle=\color{blue},
  commentstyle=\color{dkgreen},
  stringstyle=\color{mauve},
  breaklines=true,
  breakatwhitespace=true,
  tabsize=3,
}

% 轻松引用, 可以用\cref{}指令直接引用, 自动加前缀. 
% 例: 图片label为fig:1
% \cref{fig:1} => Figure.1
% \ref{fig:1}  => 1
\usepackage[capitalize]{cleveref}
% \crefname{section}{Sec.}{Secs.}
\Crefname{section}{Section}{Sections}
\Crefname{table}{Table}{Tables}
\crefname{table}{Table.}{Tabs.}

% \setmainfont{Palatino Linotype.ttf}
% \setCJKmainfont{SimHei.ttf}
% \setCJKsansfont{Songti.ttf}
% \setCJKmonofont{SimSun.ttf}
\punctstyle{kaiming}
% 偏好的几个字体, 可以根据需要自行加入字体ttf文件并调用

\renewcommand{\emph}[1]{\begin{kaishu}#1\end{kaishu}}

% 对 section 等环境的序号使用中文
\renewcommand \thesection{\zhnum{section}、}
\renewcommand \thesubsection{\arabic{section}}


%%%%%%%%%%%%%%%%%%%%%%%%%%%
%改这里可以修改实验报告表头的信息
\newcommand{\name}{张钧玮}
\newcommand{\studentNum}{2023K8009908003}
\newcommand{\major}{计算机科学与技术}
\newcommand{\labNum}{5.3}
\newcommand{\labName}{深度学习算法与硬件加速器}
%%%%%%%%%%%%%%%%%%%%%%%%%%%

\begin{document}

\input{tex_file/head.tex}

\section{支持RISCV乘法指令的硬件设计}

\noindent
$\bullet$
\textbf{booth算法的乘法运算单元设计}

% \begin{figure}[H]
%   \centering
%   \begin{minipage}[b]{0.3\linewidth}
%     \includegraphics[width=\linewidth]{fig/prj3/FSM_encoding.png}
%     \caption{FSM状态编码}
%     \label{fig:FSM_encoding}
%   \end{minipage}%
%   \hspace{5pt}
%   \begin{minipage}[b]{0.4\linewidth}
%     \includegraphics[width=\linewidth]{fig/prj3/FSM_section1.png}
%     \caption{FSM第一部分}
%     \label{fig:FSM_section1}
%   \end{minipage}
% \end{figure}

\begin{lstlisting}[language=Verilog] % 指定语言
module mul(
	input         clk,
	input         rst,
	input  [31:0] a,
	input  [31:0] b,
	input         valid,
	output        ready,
	output [63:0] result
);

localparam WAIT  = 4'b0001,
           ADD   = 4'b0010, // from Q_n and Q_{n+1} generate X or X comp to ADDER, then generate result
           SHIFT = 4'b0100, // shift AQ right one bit
           RSP   = 4'b1000;

reg [3:0] currentState, nextState;

reg  [65:0] AQ;
reg  [32:0] X;
wire [32:0] X_cmp;
reg  [ 4:0] cnt;
wire [32:0] addNum;

always @(posedge clk) begin
	if (rst)
		currentState <= WAIT;
	else
		currentState <= nextState;
end

always @(*) begin
	case (currentState)
		WAIT: begin
			if (valid)
				nextState = ADD;
			else
				nextState = WAIT;
		end
		ADD: begin
			if (cnt == 5'b0)
				nextState = RSP;
			else
				nextState = SHIFT;
		end
		SHIFT: begin
			nextState = ADD;
		end
		RSP: begin
			nextState = WAIT;
		end
		default: nextState = WAIT;
	endcase
end

/* assign registers */
always @(posedge clk) begin
	if (rst)
		X <= 33'b0;
	else if (currentState == WAIT)
		X <= {a[31], a}; // two sign bits
	else
		X <= X;
end

always @(posedge clk) begin
	if (rst)
		AQ <= 66'b0;
	else if (currentState ==  WAIT)
		AQ <= {33'd0, b, 1'b0}; // append 0 at the end
	else if (currentState == ADD)
		AQ <= {(AQ[65:33] + addNum), AQ[32:0]}; // add
	else if (currentState == SHIFT)
		AQ <= {AQ[65], AQ[65:1]}; // shift right
	else
		AQ <= AQ;
end

always @(posedge clk) begin
	if (rst)
		cnt <= 5'b0;
	else if (currentState == WAIT)
		cnt <= 5'd31;
	else if (currentState == ADD)
		cnt <= cnt - 1'b1;
	else
		cnt <= cnt;
end

assign X_cmp = ~X + 33'b1;

/* ADD state */
assign addNum = (~AQ[1] &  AQ[0]) ? X     :
                ( AQ[1] & ~AQ[0]) ? X_cmp : 33'b0;

/* RSP state */
assign ready = (currentState == RSP) ? 1'b1 : 1'b0;
assign result = AQ[65:2];

endmodule
\end{lstlisting}

\noindent
关键设计点说明:

乘法计算单元采用booth补码乘法算法,有4个状态:

$\bullet$ WAIT:等待输入valid信号拉高,并将输入的a和b存入寄存器。同时初始化counter为31,A置0,Q置b(AQ长66位,因为被乘数采用两位符号位防止溢出,同时在Q的末尾补零方便booth算法第一步计算)。接着跳转到ADD

$\bullet$ ADD:根据AQ的最后两位产生控制信号,将a或-a或0加到A上;之后将counter减一。若counter等于0,则下一步跳转到RSP,否则跳转到SHIFT

$\bullet$ SHIFT:AQ整体算数右移一位,接着跳转回ADD

$\bullet$ RSP:拉高ready信号返回计算结果,之后进入WAIT状态等待下一请求

这样保证对于每个32位输入,运行32次加法,31次移位运算(最后一次加法后不移位),从而得到正确的乘法结果


\noindent
$\bullet$
\textbf{custom_cpu中的修改}

增加MUL状态编码与状态转移:

\begin{lstlisting}[language=Verilog]
    /* define states */
	localparam INIT = 10'b0000000001,
	           IF   = 10'b0000000010,
	           IW   = 10'b0000000100,
	           ID   = 10'b0000001000,
	           EX   = 10'b0000010000,
	           ST   = 10'b0000100000,
	           LD   = 10'b0001000000,
	           RDW  = 10'b0010000000,
	           WB   = 10'b0100000000,
	           MUL  = 10'b1000000000; // add MUL state
	reg [9:0] current_state, next_state;
	
	// ...
	
	/* decode logic */
	wire is_MUL       = (is_RType & func7[0]);
	
	// ...
	
	/* FSM section 2 */
	EX:
				if (is_BType)
					next_state = IF;
				else if (is_SType)
					next_state = ST;
				else if (is_I_LDType)
					next_state = LD;
				else if (is_MUL)        // add this case
					next_state = MUL;
				else
					next_state = WB;
	MUL:
				if (mul_ready)
					next_state = WB;
				else
					next_state = MUL;
\end{lstlisting}

实例化乘法运算模块并正确连接信号

\begin{lstlisting}[language=Verilog]
    // mul
	mul mul (.clk(clk), .rst(rst), .a(mul_a), .b(mul_b), .valid(mul_valid),
	        .ready(mul_ready), .result(mul_result));
	
    assign mul_valid       = (current_state == MUL) ? 1'b1 : 1'b0;
    assign mul_a = src1;
    assign mul_b = src2;
    assign RF_wdata =  is_I_LDType            ? readData_processed
	                : (is_I_JType | is_JType) ? snpc
	                : is_LUI                  ? sext_imm
	                : is_shiftType            ? shiftOut
	                : is_MUL                  ? mulResultReg[31:0] // add this case
	                :                           aluOut;

\end{lstlisting}

这样,在接收到MUL指令后,CPU进入MUL状态由mul单元完成运算,之后进入WB阶段将结果写回寄存器,符合指令手册的要求。

\section{卷积与池化的软件算法,硬件加速器与性能计数器}

\noindent
$\bullet$
\textbf{应用滑动窗口的卷积与池化算法}。

\begin{lstlisting}[language=C]
void convolution()
{
    // ...
	//TODO: Please add your implementation here
	
	/* calculate sizes */
	short FilterSize = 1 + mul(weight_size.d2, weight_size.d3); // bias + 5 * 5 weights
	short InputSize = mul(input_fm_w, input_fm_h);
	short OutputSize = mul(conv_size.d2, conv_size.d3);

	for (short ch = 0; ch < conv_size.d1; ch++) { // for each output channel
		short bias = *weight;
		short *filter = weight + 1;

		short *channel_out = out; // output to this channel

		for (short r = 0; r < conv_out_h; r++) { // for each output row
			short r_base = mul(r, (short)stride) - pad; // base row in input
			
			for (short c = 0; c < conv_out_w; c++) { // for each output col
				short c_base = mul(c, (short)stride) - pad; // base col in input
				
				int result = 0; // store in in to avoid overflow

				short *filter_ptr = filter;
				short *input_ch = in;
				for (short n = 0; n < rd_size.d1; n++) { // for each input channel
					for (short i = 0; i < weight_size.d2; i++) { // filter rows
						short h_pos = r_base + i;

						for (short j = 0; j < weight_size.d3; j++) { // filter cols
							short w_pos = c_base + j;
							
							if (h_pos >= 0 && h_pos < input_fm_h && w_pos >= 0 && w_pos < input_fm_w) { // not in padding
								short input_val = input_ch[mul(h_pos, (short)input_fm_w) + w_pos];
								result += mul(input_val, *filter_ptr);
							}
							filter_ptr++; // since weight array is consecutive, inc directly
						}
					}
					input_ch += InputSize; // next input channel
				}

				// store result
				channel_out[mul(r, (short)conv_size.d3) + c] = (short)(result >> FRAC_BIT) + bias;
			}
		}

		weight += FilterSize; // next filter
		out += OutputSize; // next output channel
	}
}

void pooling()
{
	// ...
	//TODO: Please add your implementation here
	
	// calculate size
	short InputSize = mul(input_fm_h, input_fm_w);
	short OutputSize = mul(pool_out_h, pool_out_w);

	short *ch_input = out;
	short *ch_output = out;

	for (short ch = 0; ch < conv_size.d1; ch++) { // for each channel
		for (short r = 0; r < pool_out_h; r++) { // for each output row
			short r_base = mul(r, (short)stride) - pad;

			for (short c = 0; c < pool_out_w; c++) { // for each output col
				short c_base = mul(c, (short)stride) - pad;

				short max_val = 0x8000; // min short value

				for (short i = 0; i < KERN_ATTR_POOL_KERN_SIZE; i++) { // pool rows
					short h_pos = r_base + i;

					for (short j = 0; j < KERN_ATTR_POOL_KERN_SIZE; j++) { // pool cols
						short w_pos = c_base + j;

						if (h_pos >= 0 && h_pos < input_fm_h && w_pos >= 0 && w_pos < input_fm_w) { // not in padding
							short val = ch_input[mul(h_pos, (short)input_fm_w) + w_pos];
							if (val > max_val) max_val = val;
						}
					}
				}
				// write output
				ch_output[mul(r, (short)pool_out_w) + c] = max_val;
			}
		}
		// next channel
		ch_input += InputSize;
		ch_output += OutputSize;
	}
}
\end{lstlisting}

\noindent
$\bullet$
\textbf{硬件加速器}。

\begin{lstlisting}[language=C]
void launch_hw_accel()
{
    // ...
	//TODO: Please add your implementation here

	*gpio_start = 1;
	while (*(gpio_done) != 1);
	*gpio_start = 0;
}
\end{lstlisting}

\noindent
$\bullet$
\textbf{性能计算器设计}。

\begin{lstlisting}[language=C]
int main()
{
	Result res;
	res.msec = 0;

	bench_prepare(&res);
    
    // ...
    
    bench_done(&res);
    printf("========== Performance Counter ==========\n");
    printf("Clock cycle count: %d\n", res.msec);
    printf("=========================================\n");
    
    // ...
}
\end{lstlisting}

统计运行需要的时钟周期数,作为性能标准

\section{不同实现方法的性能差异}

性能计数器的统计结果如下:

1. 使用硬件加速器:6,832,623 cycle

2. 使用乘法单元:564,597,712 cycle

3. 用加减法代替乘法:4,176,521,680 cycle

可以看出实现MUL指令之后,软件可以在编译时使用该指令而无需用加减法算法替代乘法,从而获得一个数量级的性能提升。

若是进一步实现硬件加速器,则可以直接获得两个数量级的性能提升!

这充分说明了在特定应用场景下,有针对性地进行硬件优化的重要性。


\section{实验过程中遇到的问题、对问题的思考过程及解决方法}

\noindent
$\bullet$
\textbf{RTL设计问题}。

在编写乘法单元时犯了两个低级错误:一是移位使用了逻辑移位而非算术移位,二是最后结果的赋值应选取最高的64位,但写成了从次高位开始的64位。

由于本次实验框架没有提供行为仿真,导致在debug的时候效率低下。于是我自己撰写了用于测试乘法运算单元的c++测试用例,并用本地的verilator进行了行为仿真(因为我刚好自己也在用verilator进行一些项目,所以有一定基础)。

\begin{lstlisting}[language=C]
#include <verilated.h>
#include <verilated_vcd_c.h>
#include "Vmul.h"
#include <iostream>
#include <cstdlib>
#include <limits>

class MulTester {
    Vmul* dut;
    VerilatedVcdC* tfp;
    vluint64_t sim_time;
    bool trace_en;
    
public:
    MulTester(bool trace = false) : trace_en(trace), sim_time(0) {
        dut = new Vmul;
        Verilated::traceEverOn(trace);
        if (trace) {
            tfp = new VerilatedVcdC;
            dut->trace(tfp, 99);
            tfp->open("mul_waves.vcd");
        }
        reset();
    }

    ~MulTester() {
        if (trace_en) tfp->close();
        delete dut;
    }

    void reset() {
        dut->rst = 1;
        tick();
        dut->rst = 0;
        tick();
    }

    void tick() {
        dut->clk = 0;
        dut->eval();
        if (trace_en) tfp->dump(sim_time++);
        
        dut->clk = 1;
        dut->eval();
        if (trace_en) tfp->dump(sim_time++);
    }

    uint64_t multiply(uint32_t a, uint32_t b) {
        dut->a = a;
        dut->b = b;
        dut->valid = 1;
        
        while (!dut->ready) {
            tick();
        }
        uint64_t res = dut->result;
        
        dut->valid = 0;
        tick(); // Return to WAIT state
        
        return res;
    }
};

int main(int argc, char** argv) {
    Verilated::commandArgs(argc, argv);
    MulTester tb(true); // Enable waveform tracing
    
    // Test vectors: {a, b}
    std::vector<std::pair<int32_t, int32_t>> tests = {
        {0, 0},
        {1, 1},
        {5, 7},
        {-2, 4},
        {3, -5},
        {-4, -6},
        {0x7FFFFFFF, 1},   // MAX_INT32
        {0x80000000, 1},   // MIN_INT32
        {0x12345678, 0x9ABCDEF0}
    };
    
    for (auto& test : tests) {
        int32_t a = test.first;
        int32_t b = test.second;
        int64_t expected = static_cast<int64_t>(a) * static_cast<int64_t>(b);
        uint64_t actual = tb.multiply(static_cast<uint32_t>(a), 
                                     static_cast<uint32_t>(b));
        
        std::cout << "Test: " << a << " * " << b << " = " 
                  << "Expected: " << expected << " (" << std::hex << expected
                  << "), Got: " << actual << " (" << std::hex << actual << ")\n";
                  
        if (static_cast<int64_t>(actual) != expected) {
            std::cerr << "MISMATCH!\n";
            return 1;
        }
    }
    
    std::cout << "All tests passed!\n";
    return 0;
}
\end{lstlisting}

有着这些简易的基础设施,我很快定位了上述问题并修改通过了测试。这两个文件还可以在之后的其他选做实验中修改复用,辅助调试。

\noindent
$\bullet$
\textbf{软件算法问题}。

最开始我使用指针运算实现卷积与池化算法,但是发现在多重循环地情况下,指针算法并不简洁,同时也给调试增加了不小难度。因此,我重写了这些算法,通过记录偏移和使用mul(short a, short b)的方式记录内存访问位置,通过了测试。


\section{对讲义中思考题(如有)的理解和回答}

1. \textbf{如何避免出现乘法溢出或者精度损失?}

原数据通过short类型的定点数存放,低10位为小数部分,高6位为整数(含符号)。
通过乘法,将产生32位的数据,高12位为整数,低20位位小数。则使用\textbf{整型变量暂存}结果,便可以保证数据不溢出,且保留全部的精度。

运算结束后,需要将结果转化为原定点数类型存储起来的时候,可以首先将数据右移10位(只保留10位小数),然后截断为short类型。

\section{收获与感想}

通过本次实验,我理解了硬件乘法单元的设计原理与实现方法。在实现支持RISC-V乘法指令的Booth算法过程中,不仅掌握了补码乘法的硬件优化策略,更通过状态机设计体会到指令执行流程的精细控制。调试阶段自行搭建Verilator测试环境的经历,使我认识到完备验证体系对硬件开发的重要性,这种自主解决问题的能力对工程实践具有显著意义。

在卷积与池化算法的软件实现中,笔者通过优化内存访问模式和多维数据组织,显著提升了代码可维护性。尤为关键的是,三种实现方式的性能对比数据(硬件加速器/乘法指令/加减法替代)带来了深刻启发:专用硬件设计能在特定场景下带来数量级的性能飞跃,这为未来软硬件协同优化提供了实证依据。

实验过程中遇到的算术移位错误和结果截取偏差,促使我更严谨地审视数据通路中的位宽处理。而性能计数器的引入,则培养了量化评估优化效果的系统性思维。

\section{实验所耗时间}

\newcommand{\labTime}{9} % 请在这里填写实验所耗时间(单位:小时)

在课后,你花费了大约\underline{\makebox[5em][c]{\labTime}}小时完成此次实验。

\end{document}